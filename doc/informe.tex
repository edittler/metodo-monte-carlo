\documentclass[a4paper]{article}
\usepackage[utf8]{inputenc} % para poder usar tildes en archivos UTF-8
\usepackage[spanish]{babel} % para que comandos como \today den el resultado en castellano
%\usepackage{a4wide} % márgenes un poco más anchos que lo usual
\usepackage{fullpage}
\usepackage{amsmath}
\usepackage{amssymb}
\usepackage[usenames,dvipsnames,svgnames,table]{xcolor} % Colores
\usepackage{graphicx} % Permite incluir gráficos
\usepackage[hang, bf]{caption} % Personaliza los subtítulos de las figuras y tablas
\usepackage{float} % Permite posicionar mejor las figuras y tablas
\usepackage{listings} % Permite insertar código fuente
\usepackage{pgfplots}
%\pgfplotsset{compat=1.13}

\definecolor{darkblue}{rgb}{0,0,0.4}
\definecolor{darkgreen}{rgb}{0,0.4,0}

\lstset{ % Elijo formato de bloques de código fuente
	backgroundcolor=\color{gray!5},
	basicstyle=\ttfamily\footnotesize,
	breaklines=true,
	commentstyle=\color{darkgreen},
	extendedchars=true,
	frame=single,
	language=octave,
	literate={á}{{\'a}}1 {é}{{\'e}}1 {í}{{\'i}}1 {ó}{{\'o}}1 {ú}{{\'u}}1 {ñ}{{\~n}}1, % Escapeo caracteres especiales
	keywordstyle=\color{darkblue},
	numbers=left,
	numberstyle=\tiny\color{gray},
	tabsize=4,
	showspaces=false,
	showstringspaces=false,
	stringstyle=\color{red}
}

\begin{document}

\title{Método Monte Carlo}
\date{\today}
\author{Ezequiel Pérez Dittler, FIUBA}

\maketitle
\newpage

\section{Enunciado}
Sea $f: \mathbb{R} \to \mathbb{R}$ una función continua y no negativa.
Sea $M > 0$ el valor máximo de la función $f$ sobre el intervalo $[a, b]$. \\

(a) Se elige al azar un punto de coordenadas $(X, Y)$ dentro del rectángulo de
vértices $(a, 0)$, $(b, 0)$, $(b, M)$, $(a, M)$. Relacionar la probabilidad
del evento $A = \{ Y \leqslant f(X) \}$ con el valor de la integral
$\int\limits_a^b f(x) dx$. \\

(b) Si se conoce el valor de la probabilidad, $P(A)$, del evento
$A = \{ Y \leqslant f(X) \}$, ¿cómo se calcula la integral
$\int\limits_a^b f(x) dx$? \\

(c) Obtener un método que permita estimar el valor de la integral
$\int\limits_a^b f(x) dx$ en base a los resultados de $n$ simulaciones del
experimento descrito en Inciso (a). \\

(d) Estimar el valor de la integral $\int\limits_0^2 e^{-x^{2}} dx$ utilizando
el método obtenido en Inciso (c) basándose en los resultados de 10000
simulaciones.

\section{Resolución}
\subsection{Inciso (a)}
\begin{figure}[h]
  \centering
  \pgfplotsset{compat=1.3}
  \begin{tikzpicture}
    \begin{axis}[
        xlabel={$X$},
		ylabel={$Y$},
        axis x line=bottom,
        axis y line=left,
		enlarge x limits=true,
		enlarge y limits=true,
		symbolic x coords={0,a,b},
		area style
      ]
      \addplot coordinates {(a,11) (b,11)} \closedcycle;
    \end{axis}
  \end{tikzpicture}
  \caption{Rectángulo de vértices $(a, 0)$, $(b, 0)$, $(b, M)$, $(a, M)$}
\end{figure}

\[
  P(A) = P( Y \leqslant f(X) ) = \frac{ area(A) }{ area(\Omega) } =
  \frac{ \int\limits_a^b f(x) dx }{ (b-a) . M }
\]

\subsection{Inciso (b)}

Siendo $P(A)$ conocido, se puede estimar el valor de la integral
$\int\limits_a^b f(x) dx$ del siguiente modo

\[
  \int\limits_a^b f(x) dx = P(A).(b-a) . M
\]

\subsection{Inciso (c)}

\begin{lstlisting}[caption={distribucionMixta.m},label={lst:distribucionMixta}]
function r = distribucionMixta(n)
  m = n * 3;
  if (m < 1)
    r = -2;
  elseif (m < 2)
    r = m * 2 - 3;
  else
    r = 2;
  end
end
\end{lstlisting}

\end{document}